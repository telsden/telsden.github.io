%%%%%%%%%%%%%%%%%%%%%%%%%%%%%%%%%%%%%%%%%
% Cies Resume/CV
% LaTeX Template
% Version 1.0 (10/2/14)
%
% This template has been downloaded from:
% http://www.LaTeXTemplates.com
%
% Original author:
% Cies Breijs (cies@kde.nl)
% https://github.com/cies/resume with extensive modifications by:
% Vel (vel@latextemplates.com)
%
% License:
% CC BY-NC-SA 3.0 (http://creativecommons.org/licenses/by-nc-sa/3.0/)
%
%%%%%%%%%%%%%%%%%%%%%%%%%%%%%%%%%%%%%%%%%

%----------------------------------------------------------------------------------------
%	PACKAGES AND OTHER DOCUMENT CONFIGURATIONS
%----------------------------------------------------------------------------------------

\documentclass[11pt,a4paper]{article} % Font size (10-12pt) and paper size (a4paper, letterpaper, legalpaper, etc)

% Copyright (c) 2012 Cies Breijs
%
% The MIT License
%
% Permission is hereby granted, free of charge, to any person obtaining a copy
% of this software and associated documentation files (the "Software"), to deal
% in the Software without restriction, including without limitation the rights
% to use, copy, modify, merge, publish, distribute, sublicense, and/or sell
% copies of the Software, and to permit persons to whom the Software is
% furnished to do so, subject to the following conditions:
%
% The above copyright notice and this permission notice shall be included in
% all copies or substantial portions of the Software.
%
% THE SOFTWARE IS PROVIDED "AS IS", WITHOUT WARRANTY OF ANY KIND, EXPRESS OR
% IMPLIED, INCLUDING BUT NOT LIMITED TO THE WARRANTIES OF MERCHANTABILITY,
% FITNESS FOR A PARTICULAR PURPOSE AND NONINFRINGEMENT. IN NO EVENT SHALL THE
% AUTHORS OR COPYRIGHT HOLDERS BE LIABLE FOR ANY CLAIM, DAMAGES OR OTHER
% LIABILITY, WHETHER IN AN ACTION OF CONTRACT, TORT OR OTHERWISE, ARISING FROM,
% OUT OF OR IN CONNECTION WITH THE SOFTWARE OR THE USE OR OTHER DEALINGS IN THE
% SOFTWARE.

%%% LOAD AND SETUP PACKAGES

\usepackage[margin=0.75in]{geometry} % Adjusts the margins

\usepackage{multicol} % Required for multiple columns of text

\usepackage{mdwlist} % Required to fine tune lists with a inline headings and indented content

\usepackage{relsize} % Required for the \textscale command for custom small caps text

\usepackage[pdftex]{hyperref} % Required for customizing links
\usepackage{xcolor} % Required for specifying custom colors
\definecolor{dark-blue}{rgb}{0.15,0.15,0.4} % Defines the dark blue color used for links
\hypersetup{colorlinks,linkcolor={dark-blue},citecolor={dark-blue},urlcolor={dark-blue}} % Assigns the dark blue color to all links in the template

\usepackage{tgpagella} % Use the TeX Gyre Pagella font throughout the document
\usepackage[T1]{fontenc}
\usepackage{microtype} % Slightly tweaks character and word spacings for better typography

\pagestyle{empty} % Stop page numbering

%----------------------------------------------------------------------------------------
%	DEFINE STRUCTURAL COMMANDS
%----------------------------------------------------------------------------------------

\newenvironment{indentsection} % Defines the indentsection environment which indents text in sections titles
{\begin{list}{}{\setlength{\leftmargin}{\newparindent}\setlength{\parsep}{0pt}\setlength{\parskip}{0pt}\setlength{\itemsep}{0pt}\setlength{\topsep}{0pt}}}{\end{list}}

\newcommand*\maintitle[2]{\noindent{\huge \textbf{#1}}\ \ \ \emph{#2}\vspace{0.3em}} % Main title (name) with date of birth or subtitle

\newcommand*\roottitle[1]{\subsection*{\Large{#1}}\vspace{-0.3em}\nopagebreak[4]} % Top level sections in the template

\newcommand{\headedsection}[3]{\nopagebreak[4]\begin{indentsection}\item[]\textscale{1.1}{#1}\hfill#2#3\end{indentsection}\nopagebreak[4]} % Section title used for a new employer

\newcommand{\headedsubsection}[3]{\nopagebreak[4]\begin{indentsection}\item[]\textscale{1.1}{\textbf{#1}}\hfill\emph{#2}#3\end{indentsection}\nopagebreak[4]} % Section title used for a new position

\newcommand{\bodytext}[1]{\nopagebreak[4]\begin{indentsection}\item[]#1\end{indentsection}\pagebreak[2]} % Body text (indented)

\newcommand{\inlineheadsection}[2]{\begin{basedescript}{\setlength{\leftmargin}{\doubleparindent}}\item[\hspace{\newparindent}\textbf{#1}]#2\end{basedescript}\vspace{-1.7em}} % Section title where body text starts immediately after the title

\newcommand*\acr[1]{\textscale{.95}{#1}} % Custom acronyms command

%\newcommand*\bull{\ \ \raisebox{-0.365em}[-1em][-1em]{\textscale{4}{$\cdot$}} \ } %
\newcommand*\bull{\raisebox{-0.365em}[-1em][-1em]{\textscale{4}{$\cdot$}} \ }  %Custom bullet point for separating content

\newlength{\newparindent} % It seems not to work when simply using \parindent...
\addtolength{\newparindent}{\parindent}

\newlength{\doubleparindent} % A double \parindent...
\addtolength{\doubleparindent}{\parindent}

\newcommand{\breakvspace}[1]{\pagebreak[2]\vspace{#1}\pagebreak[2]} % A custom vspace command with custom before and after spacing lengths
\newcommand{\nobreakvspace}[1]{\nopagebreak[4]\vspace{#1}\nopagebreak[4]} % A custom vspace command with custom before and after spacing lengths that do not break the page

\newcommand{\spacedhrule}[2]{\breakvspace{#1}\hrule\nobreakvspace{#2}} % Defines a horizontal line with some vertical space before and after it % Include structure.tex which contains packages and document layout definitions

\hyphenation{Some-long-word} % Specify custom hyphenation points in words with dashes where you would like hyphenation to occur, or alternatively, don't put any dashes in a word to stop hyphenation altogether

\usepackage{soul}
\usepackage{enumitem}
\usepackage{etaremune}
\usepackage{expdlist}

\newcommand\vs{\vspace{-0.25cm}}
%\newcommand\vp{\vspace{-0.5cm}}

\begin{document} 

%----------------------------------------------------------------------------------------
%	NAME AND CONTACT INFORMATION
%----------------------------------------------------------------------------------------

\maintitle{CV - Tom Elsden}{}  % Your name and date of birth or subtitle

\vspace{0.25cm}

\noindent Email: \href{mailto:te55@st-andrews.ac.uk}{te55@st-andrews.ac.uk} %\bull \noindent\href{http://www-solar.mcs.st-and.ac.uk/~telsden/}{Website} \\% Your email address
\\
Address: School of Mathematics and Statistics, University of St Andrews, St Andrews, UK, KY16 9SS % Your address
\\ 
ORCID: \href{https://orcid.org/0000-0002-1910-2010}{https://orcid.org/0000-0002-1910-2010} \\
Personal webpage: \href{https://telsden.github.io/}{https://telsden.github.io/} \\
%CV last updated: 17/01/2025

%\vspace{0.1cm}
%\noindent This front page provides a concise form of my CV. Details e.g. publication list, follow in later pages.
\vs

\spacedhrule{0.9em}{-0.4em} % Horizontal rule - the first bracket is whitespace before and the second is after

%----------------------------------------------------------------------------------------
%	SUMMARY SECTION
%----------------------------------------------------------------------------------------

\roottitle{Employment \& Education} % Root section title

\begin{itemize}
	\item \textit{Sep 2022 - Present}: Lecturer in Mathematics and Statistics, University of St Andrews.
	\vspace{-0.25cm}
	\item \textit{Oct 2021 - Aug 2022}: Leverhulme Early Career Research Fellow and Rankin-Sneddon Fellow, University of Glasgow. 
	\vspace{-0.25cm}
	\item \textit{Oct 2019 - Sep 2021}: Leverhulme Early Career Research Fellow, University of Leicester.
	\vspace{-0.25cm}
	\item \textit{June 2016 - Sep 2019}: Post Doctoral Research Assistant on a Leverhulme Research Project Grant, University of St Andrews.
	\vspace{-0.25cm}
	\item \textit{Sep 2012 - June 2016}: PhD in Applied Mathematics, University of St Andrews. 
	\vspace{-0.25cm}
	\item \textit{June 2011 - Aug 2011}: Research Experience for Undergraduates (REU) Programme, Harvard Smithsonian Centre for Astrophysics.
	\vspace{-0.25cm}
	\item \textit{Sep 2008 - June 2012}: MMath (Hons) Mathematics - First Class, University of St Andrews.
\end{itemize}

\spacedhrule{0.5em}{-0.4em} % Horizontal rule - the first bracket is whitespace before and the second is after

\roottitle{Research Area and Activity}

My research focusses on large scale plasma waves, known as ultra-low frequency (ULF) waves, in Earth's magnetosphere - the space around the Earth dominated by the Earth's magnetic field. These waves are an important aspect of space weather, driving aurora and radiation in near-Earth space. I use computational magnetohydrodynamic (MHD) modelling to understand and predict the role of ULF waves in Earth's magnetospheric system.

\vspace{0.1cm}
\noindent I have written 30 research papers (27 published, 1 accepted, 2 in review), with 12 as first author and 8 as second author, in leading journals for my field such as:

\begin{itemize}
	\item Journal of Geophysical Research (JGR): Space Physics (18),
		\vspace{-0.25cm}
	\item Geophysical Research Letters (GRL)(3),
		\vspace{-0.25cm}
	\item Frontiers in Astronomy and Space Science (4),   
		\vspace{-0.25cm}
	\item Invited single author review to appear in Oxford Research Encyclopedia of Planetary Science,
		\vspace{-0.25cm}
	\item Invited book chapter in AGU Geophysical Monograph Series.
\end{itemize}

%\vspace{0.1cm} 
\noindent I have delivered over 40 external research talks, over 20 of which have been invited talks, including at the American Geophysical Union annual meetings, the largest conference for my discipline. I have regularly organised sessions at national and international conferences, as well as co-organising the largest UK PhD oriented summer school in my field, the STFC Introductory Course in Solar and Solar-Terrestrial Physics, in 2023: \href{https://solar-mcs.wp.st-andrews.ac.uk/teaching/stfc-introductory-course-in-solar-and-solar-terrestrial-physics/}{link}. I have 281 citations and an h-index of 10 (Google scholar).

%My research involves the computational modelling of ultra-low frequency (ULF) waves in Earth magnetosphere - the space around the Earth dominated by the effects of Earth's magnetic field. These waves generate aurora by driving strong magnetic field-aligned currents and energise trapped particles which can be a hazardous form of radiation for orbiting spacecraft. I design and code computational models to solve the equations of magnetohydrodynamics (MHD) to model and therefore physically understand the effect of ULF waves on the near-Earth space environment.


%Make a short 1-2 page cv at start, then follow up with details of publications etc after (follow Jon structure).
%Include full talk lists etc at end.
%Include PhD supervision, and examinations.
%Include Modules taught and feedback table (see Jon)

\spacedhrule{0.9em}{-0.4em} % Horizontal rule - the first bracket is whitespace before and the second is after

\roottitle{Grants, Appointments and Memberships}

\begin{itemize}
\item 2019 - 2022, Leverhulme Early Career Fellowship entitled \textit{Resonating Magnetic Field Lines: A Process for Energy Transfer at Earth/Mercury}, University of Leicester, £93000. 
\vspace{-0.25cm} 
\item Co-I on STFC Knowledge Exchange Institutional Award, 2023, entitled \textit{Sharing the History and Science of Total Solar Eclipses}, £16000.
\vspace{-0.25cm} 
\item UK liaison to the US Geophysical Environment Modelling (GEM) community.
\vspace{-0.25cm} 
\item Core member of 2 International Space Science Institute (ISSI) teams from 2019-2024. 
\vspace{-0.25cm} 
\item Lead of public outreach activities in mathematics and statistics, University of St Andrews. 
\vspace{-0.25cm} 
\item Fellow of the Royal Astronomical Society.
\vspace{-0.25cm} 
\item Member of the American Geophysical Union (AGU).
\vspace{-0.25cm} 
\item Regular reviewer (2-3 per year) for journals JGR Space Physics, GRL and Frontiers in Astronomy and Space Science and grant applications for STFC.
\end{itemize}

\spacedhrule{-0.2em}{-0.4em} % Horizontal rule - the first bracket is whitespace before and the second is after

\roottitle{Teaching and Supervision}

I have lectured several courses while at St Andrews. Below I list these modules, where * means that I was the module coordinator, together with an average feedback score from student questionnaires on a 1 (good) to 5 (bad) scale. Students are asked questions about the lecturer specifically as to whether lectures were well organised, well explained, well presented and could they contact me if needed.
%\begin{itemize}
%	\item MT1002 Mathematics, Sem 1 2024/25 (St Andrews)
%	\item MT2503 Multivariate Calculus*, Sem 1 2023/24 (St Andrews)
%	\item MT4112 Computing in Mathematics, 12 lectures, Sem1 2017/2018 (St Andrews)
%	\item MT4507 Classical Mechanics* (St Andrews)
%	\item MT4510 Solar Theory*, Sem 2 2023/24 \& 2024/2025 (St Andrews)
%	\item Mathematics 1C, 12 lectures, Sem 1 2021/2022 (Glasgow)
%	\item Mechanics 2E, 12 lectures, Sem 2 2021/2022 (Glasgow) 
%\end{itemize}

\begin{center}
	\begin{tabular}{|c|c|c|c|c|c|}
		\hline
		Year & Semester & Module & Class Size & Average Score (1 to 5) \\ 
		\hline
		2022/2023 & 2 & MT4507 Classical Mechanics* & 26 & 1.13 \\ 
		\hline
		2023/2024 & 1 & MT2503 Multivariate Calculus* & 238 & 1.29 \\ 
		\hline
		2023/2024 & 2 & MT4510 Solar Theory* & 29 &  1.07 \\ 
		\hline
		2024/2025 & 1 & MT1002 Mathematics & 264 &  1.21 \\ 
		\hline
		
	\end{tabular}
\end{center}

\noindent As a post doctoral researcher at the University of Glasgow, I lectured on the modules Mathematics 1C and Mechanics 2E in the 2021/2022 academic year.

\vs
\begin{itemize}
	\item I am currently the primary supervisor for one PhD student, James Brooks, who started in September 2024, funded by an STFC Doctoral Training Partnership (DTP) grant, ST/Y509577/1, to the solar and magnetospheric research group at St Andrews.
	
	\vs
	\item I am second supervisor (10\%) to one further PhD student, Rachel Davies, due to finish this year.
	
	\vs
	\item In summer 2023 I supervised 6 undergraduate summer students, on a STFC funded knowledge exchange project to create a website sharing the history and science of total solar eclipses \href{https://eclipse-history.wp.st-andrews.ac.uk/}(link).
	\vs
	\item I have supervised a further summer student and 5 undergraduate final year projects (2 Masters, 3 BSc) at St Andrews.  
	\vs
	\item Internal viva examiner for the PhD thesis of Kate Mowbray on 15/01/2025.
\end{itemize}

\vs
\spacedhrule{0.5em}{-0.4em} % Horizontal rule - the first bracket is whitespace before and the second is after

\roottitle{Publications}


\begin{etaremune}	%[leftmargin=0cm,itemindent=.5cm,labelwidth=\itemindent,labelsep=0cm,align=left]


\item Adnane Osmane, Jasmine Sandhu, \underline{\textbf{Tom Elsden}}, Oliver Allanson, Lucile Turc, Radial Diffusion Driven by Spatially Localized ULF Waves in the Earth's Magnetosphere, \textit{submitted} to GRL.

\vs
\item
Wright, A. N., \underline{\textbf{Elsden, T.}}, Degeling, A., Mann, I. R., Ozeke, L., Yeoman, T., Sandhu, J., Takahashi, T., Poloidal Field Line Resonances Driven by a Fast Wave, \textit{submitted} to GRL.

\vs
\item \underline{\textbf{Elsden, T.}}, Ultra Low Frequency Waves of Earth's Magnetosphere – Review Article, (\textit{accepted} by Oxford Research Encyclopedia of Planetary Science).

\vs
\item Archer MO, Pilipenko VA, Li B, Sorathia K, Nakariakov VM, \underline{\textbf{Elsden T}} and Nykyri K (2024) Magnetopause MHD surface wave theory: progress \& challenges. Front. Astron. Space Sci. 11:1407172. \href{https://doi.org/10.3389/fspas.2024.1407172}{doi: 10.3389/fspas.2024.1407172}
\vs

\item Wright, A.N., Hartinger, M.D., Takahashi, K. and  \underline{\textbf{Elsden, T.}} (2024). Alfvén Waves in the Earth's Magnetosphere. In Alfvén Waves Across Heliophysics, A. Keiling (Ed.).
\newline 
\href{https://doi.org/10.1002/9781394195985.ch10}{doi:10.1002/9781394195985.ch10}
\vs 

\item Allanson O, Ma D, Osmane A, Albert JM, Bortnik J, Watt CEJ, Chapman SC, Spencer J, Ratliff DJ, Meredith NP, \underline{\textbf{Elsden T}}, Neukirch T, Hartley DP, Black R, Watkins NW and Elvidge S (2024) The challenge to understand the zoo of particle transport regimes during resonant wave-particle interactions for given survey-mode wave spectra. Front. Astron. Space Sci. 11:1332931. 
\newline \href{https://doi.org/10.3389/fspas.2024.1332931}{doi:10.3389/fspas.2024.1332931}

\vs
\item Sandhu, J. K., Degeling, A. W., \underline{\textbf{Elsden, T.}}, Murphy, K. R., Rae, I. J., Wright, A. N., et al. (2023). Van Allen Probes observations of a three-dimensional field line resonance at a plasmaspheric plume. GRL, 50, 
\href{https://doi.org/10.1029/2023GL106715}{doi:10.1029/2023GL106715}
\vs


\item Hartinger, M. D., \underline{\textbf{Elsden, T.}}, Archer, M. O., Takahashi, K., Wright, A. N., Artemyev, A., et al. (2023). Properties of Magnetohydrodynamic normal modes in the Earth's Magnetosphere. JGR: Space Physics, 128, \href{https://doi.org/10.1029/2023JA031987}{doi:10.1029/2023JA031987}

\vs
\item Takahashi, K., \underline{\textbf{Elsden, T.}}, Wright, A. N., \& Degeling, A. W. (2023). Polarization of magnetospheric ULF waves excited by an interplanetary shock on 27 February 2014. JGR: Space Physics, 128. \href{https://doi.org/10.1029/2023JA031608}{doi:10.1029/2023JA031608}

\vs
\item Wright, A., \& \underline{\textbf{Elsden, T.}} (2023). Resonant Fast-Alfvén Wave Coupling in a 3D Coronal Arcade. Physics, 5(1), 310-321. \href{https://doi.org/10.3390/physics5010023}{doi:10.3390/physics5010023}

\vs
\item \underline{\textbf{Elsden, T.}} \& Southwood, D. J. (2023). Modeling features of field line resonance observable by a single spacecraft at Saturn. JGR: Space Physics, 128, \href{https://doi.org/10.1029/2022JA031208}{doi:10.1029/2022JA031208}

\vs
\item Fogg, A. R., Lester, M., Yeoman, T. K., Carter, J. A., Milan, S. E., Sangha, H. K.,  \underline{\textbf{Elsden, T.}} et al. (2023). Multi-instrument observations of the effects of a solar wind pressure pulse on the high latitude ionosphere: A detailed case study of a geomagnetic sudden impulse. JGR: Space Physics, 128, \href{https://doi.org/10.1029/2022JA031136}{doi:10.1029/2022JA031136}

\vs
\item \underline{\textbf{Tom Elsden}}, Matthew K James, Jasmine K Sandhu, Clare Watt, RAS Specialist Discussion Meeting Report, Astronomy \& Geophysics, Volume 63, Issue 5, October 2022, Pages 5.26–5.30, \newline \href{https://doi.org/10.1093/astrogeo/atac066}{doi.org/10.1093/astrogeo/atac066}

\vs
\item  \underline{\textbf{Elsden, T.}}, Wright A and Degeling A (2022) A review of the theory of 3‐D Alfvén (field line) resonances. Front. Astron. Space Sci. 9:917817. \href{https://doi.org/10.3389/fspas.2022.917817}{doi: 10.3389/fspas.2022.917817}

\vs
\item Sakurai, T., Wright, A. N., Takahashi, K., \underline{\textbf{Elsden, T.}}, Ebihara, Y., Sato, N., et al. (2022). Poleward moving auroral arcs and Pc5 oscillations. JGR: Space Physics, 127, \href{https://doi.org/10.1029/2022JA030362}{doi:10.1029/2022JA030362}

\vs
\item Wright, A. N., Degeling, A., \underline{\textbf{Elsden, T.}} (2022). Resonance Maps for 3D Alfv\'{e}n Waves in a Compressed Dipole Field, JGR: Space Physics, 127, \href{https://doi.org/10.1029/2022JA030294}{10.1029/2022JA030294}. 

\vs
\item \underline{\textbf{Elsden, T.}}, Wright, A. N. (2022). Polarization Properties of 3-D Field Line Resonances, JGR: Space Physics, 127, \href{https://doi.org/10.1029/2021JA030080}{doi:10.1029/2021JA030080}.

\vs
\item \underline{\textbf{Elsden, T.}}, Yeoman, T.K., Wharton, S. J., Rae, I. J., Sandhu, J. K., Walach, M.-T., James, M. K., Wright, D. M. (2022). Modeling the Varying Location of Field Line Resonances During Geomagnetic Storms, JGR: Space Physics, 127, \href{https://doi.org/10.1029/2021JA029804}{doi:10.1029/2021JA029804}.

\vs
\item Allanson, O., \underline{\textbf{Elsden T.}}, Watt, C., Neukirch, T. (2022). Weak Turbulence and Quasilinear Diffusion for Relativistic Wave-Particle Interactions via a Markov Approach. Frontiers in Astronomy and Space Sciences, 14 January 2022. \href{https://doi.org/10.3389/fspas.2021.805699}{doi:10.3389/fspas.2021.805699}.  

\vs
\item Sandhu, J. K., Rae, I. J., Staples, F. A., Hartley, D. P., Walach, M.-T., \underline{\textbf{Elsden, T.}}, Murphy, K. R. (2021). The roles of the magnetopause and plasmapause in storm-time ULF wave power enhancements. JGR: Space Physics, 126, \href{https://doi.org/10.1029/2021JA029337}{doi:10.1029/2021JA029337}

\vs
\item Southwood, D. J., Cao, H., Shebanits, O., \underline{\textbf{Elsden, T.}}, Hunt, G., Dougherty, M. (2021), Discovery of Alfv\'{e}n Waves Planetward of Saturn's Rings, JGR: Space Physics, 125, {\href{https://doi.org/10.1029/2020JA028473}{doi:10.1029/2020JA028473}}.

\vs
\item \underline{\textbf{Elsden, T.}}, A. N. Wright, (2020), Evolution of High-m Poloidal Alfv\'{e}n Waves in a Dipole Magnetic Field, JGR: Space Physics, 125, {\href{https://doi.org/10.1029/2020JA028187}{doi:10.1029/2020JA028187}}.

\vs
\item Wright, A. N., \underline{\textbf{Elsden, T.}} (2020), Simulations of MHD wave propagation and coupling in a 3‐D magnetosphere, JGR: Space Physics, 125,  {\href{https://doi.org/10.1029/2019JA027589}{doi:10.1029/2019JA027589}}.

\vs
\item \underline{\textbf{Elsden, T.}}, A. N. Wright, (2019), The Effect of Fast Normal Mode Structure and Magnetopause Forcing on FLRs in a 3D Waveguide, JGR Space Physics, 124, {\href{https://doi.org/10.1029/2018JA026222}{doi:10.1029/2018JA026222}}.

\vs
\item Wright, A. N., \underline{\textbf{Elsden, T.}}, K. Takahashi, (2018) Modeling the Dawn/Dusk Asymmetry of Field Line Resonances, JGR Space Physics, 123, {\href{https://doi.org/10.1029/2018JA025638}{doi:10.1029/2018JA025638}}.

\vs
\item \underline{\textbf{Elsden, T.}}, A. N. Wright (2018),  The Broadband Excitation of 3D Alfv\'{e}n Resonances in a MHD Waveguide, JGR Space Physics, 123, {\href{https://doi.org/10.1002/2017JA025018}{doi:10.1002/2017JA025018}}.

\vs
\item \underline{\textbf{Elsden, T.}}, A. N. Wright (2017), The Theoretical Foundation of 3D Alfv\'{e}n Resonances: Time Dependent Solutions, JGR Space Physics, 122, {\href{http://onlinelibrary.wiley.com/doi/10.1002/2016JA023811/full}{doi:10.1002/2016JA023811}}.

\vs
\item Wright, A. N., \underline{\textbf{Elsden, T.}} (2016), The Theoretical Foundation of 3D Alfv\'{e}n Resonances: Normal Modes, Astrophysical Journal, 833, 230, doi:{\href{http://iopscience.iop.org/article/10.3847/1538-4357/833/2/230}{10.3847/1538-4357/833/2/230 }}.

\vs
\item \underline{\textbf{Elsden, T.}}, A. N. Wright, M. D. Hartinger (2016), Deciphering satellite observations of compressional ULF waveguide modes, JGR Space Physics, 121, \newline doi:{\href{http://onlinelibrary.wiley.com/doi/10.1002/2016JA022351/full}{10.1002/2016JA022351}}.

\vs
\item \underline{\textbf{Elsden, T.}}, A. N. Wright (2015), The use of the Poynting vector in interpreting ULF waves in magnetospheric waveguides, JGR Space Physics, 120, doi:{\href{http://onlinelibrary.wiley.com/doi/10.1002/2014JA020748/abstract}{10.1002/2014JA020748}}.

\end{etaremune}

%\noindent Saar, Steven H., Elsden, T., Muglack, K. (2012), An Exploration of the Emission Properties of X-ray Bright Points Seen With SDO, American Astronomical Society, AAS Meeting 220. {\href{http://adsabs.harvard.edu/abs/2012AAS...22020713S}{Abstract}

%\vspace{1.5em}



%\roottitle{Professional Activities}
%Conference organisation, GEM liaison etc. 

\end{document}

%----------------------------------------------------------------------------------------












%----------------------------------------------------------------------------------------
\roottitle{Key Experience}
%\headedsection
%{Teaching - University of St Andrews}
%{\textsc{St Andrews, Scotland}}

\vspace{0.2cm}

\headedsubsection{\underline{Grants}}{}{}

{\bodytext{

\bull Leverhulme Early Career Fellowship, held at the University of Leicester (Oct 2019 - Sep 2021) and the University of Glasgow (Oct 2021 - Sep 2022), £93000.

\vspace{0.1cm}

\bull Institute of Mathematics and its Applications QJMAM conference travel fund, April 2019, £400.

}}

\vspace{0.2cm}

\headedsubsection{\underline{Lecturing and Tutoring}}{}{}

\inlineheadsection{University of Glasgow}{} 
\vspace{0.3cm}
{\bodytext{
\bull Semester 2, 2021/2022, Lecturer Mechanics 2E - second year mathematics course. Responsible for delivering half of the lecture material through recorded online lectures as well as setting part of the final exam. 

\vspace{0.1cm}
\bull Semester 1, 2021/2022, Lecturer Mathematics 1C - first year introductory mathematics course. Online lectures using Zoom, Microsoft One Note and Microsoft Teams. Online continuous assessment prior to each lecture provided through WebAssign quizzes. 
}}

\inlineheadsection{University of St Andrews}{} 
\vspace{0.3cm}
\bodytext{\bull Lecturer for 4th year undergraduate course MT4112 - Computing in Mathematics, Semester 1, 2017. This involved setting and marking coursework and the final exam. Received excellent feedback on the module evaluation questionnaire, with an overall score of 1.38 on a 1 (good) to 5 (bad) scale.}

\vspace{0.1cm}

\bodytext{\bull Undergraduate tutor for multiple courses in mathematics such as introductory 1st and 2nd year mathematics, applied mathematics, computing with Python and multivariate calculus. Assisted in the running of several large computing `labs' at sub-honours level.

}

\vspace{0.2cm}
\headedsubsection
{\underline{Conference Session Organisation}}{}

\bodytext{
\bull Co-organiser, session: `Observations and modelling of the effects of solar wind pressure pulses on the terrestrial magnetosphere', European Geophysical Union (EGU) General Assembly, 23rd-27th May, 2022.

\bull Lead organiser: `Planetary Ultra-Low Frequency Waves: Theory, Modelling and Observations', Royal Astronomical Society Specialist Discussion Meetings, 11th March 2022: \href{https://www.youtube.com/watch?v=qrz3cYI5mDQ&t=1150s&ab_channel=RoyalAstronomicalSociety}{YouTube link}.
\vspace{0.1cm}

\bull Co-organiser: `Space weather and plasma processes: From the Sun to the Earth', National Astronomy Meeting, Bath, UK (virtual), 19th-23rd July 2021. 
\vspace{0.1cm}

\bull Lead session organiser: `Connecting MHD Wave Research from the Sun to the Magnetospheres', National Astronomy Meeting, Lancaster, UK, 30 June - 4 July 2019.
\vspace{0.1cm}

\bull Invited session organiser and speaker, Geospace Environment Modelling (GEM), Santa Fe, NM, USA, 22-28 June 2019.
\vspace{0.1cm}
}

\vspace{0.2cm}
\headedsubsection
{\underline{Invited Talks}}{}

\bodytext{

%\bull Invited review talk on `Alfv\'{e}n Waves in the Magnetosphere', AGU Chapman Conference on Alfv\'{e}n Waves (delayed until 2023).
%\vspace{0.1cm}

\bull University of Dundee, Mathematics Seminar - 24th Jan 2022.
\vspace{0.1cm}

\bull Magnetosphere Online Seminar Series, organised by American Geospace Environment Modelling community, 16th Aug 2021, available on Youtube \href{https://www.youtube.com/watch?v=hp-VedV-6eU&t=3158s&ab_channel=MagnetosphereSeminars}{here}. 
\vspace{0.1cm}

\bull Northumbria University, Solar/Space Physics Research Group - 15th Jun 2021.
\vspace{0.1cm}

\bull Imperial College London, Space Physics Group Seminar - 18th Jun 2020.
\vspace{0.1cm}

\bull University of St Andrews, Solar Group Seminar - 12th Feb 2020.
\vspace{0.1cm}

\bull American Geophysical Union Fall Meeting, San Francisco, USA - 9th Dec 2019. 
\vspace{0.1cm}

\bull University of Warwick, Centre for Fusion, Space and Astrophysics Seminar - 7th Feb 2019. \vspace{0.1cm}

\bull University College London, Mullard Space Science Lab (MSSL), Space Plasma Physics Group Seminar - 9th Oct 2018. \vspace{0.1cm}

\bull University of Leicester, Radio Space Plasma Physics Group Seminar - 26th Sep 2018. \vspace{0.1cm}

\bull STFC Advanced Summer School in Solar, Solar-Terrestrial and Solar-Planetary Physics, University of Southampton, talk on MHD Waves - 11th Sep 2018. \vspace{0.1cm}

\bull University of Dundee, Mathematics Seminar - 12th Mar 2018. \vspace{0.1cm}

\bull University of Glasgow, Astronomy Seminar - 1st Feb 2018. \vspace{0.1cm}

\bull Russian-British Seminar of Young Scientists, Irkutsk, Russia - 18th Sep 2017. 
\vspace{0.1cm}

\bull University of Cambridge, Department of Applied Mathematics and Theoretical Physics, Astronomy Seminar - 8th May 2017.

\vspace{0.1cm}
}

\vspace{0.2cm}
\headedsubsection
{\underline{Memberships and Professional Roles}}{}

{\vspace{0.1cm} \bodytext{

\bull UK liaison to the US Geophysical Environment Modelling (GEM) community. This involves writing a yearly report on relevant UK research for dissemination to GEM researchers worldwide.
\vspace{0.1cm}

\bull Core member of the International Space Science Institute (ISSI) team investigating 3D Alfv\'{e}n resonances (Aug 2019 - Aug 2022).\vspace{0.1cm}

\bull Core member of the application for the International Space Science Institute (ISSI) team on 'Magnetohydrodynamic surface waves at Earth’s magnetosphere (and beyond)'. 
\vspace{0.1cm}

\bull Reviewer for \textit{JGR: Space Physics}, \textit{Nature Communications}, \textit{GRL}, \textit{Frontiers in Astronomy and Space Sciences} and \textit{Earth, Planets and Space}.

\vspace{0.1cm}
\bull Fellow of the Royal Astronomical Society.

\vspace{0.1cm}
\bull Member of the American Geophysical Union.

\vspace{0.1cm}
\bull Honorary Research Fellow, University of St Andrews, Nov 2019 - Aug 2022. 
}}

\vspace{0.2cm}

\vspace{1.5em}
\spacedhrule{-0.2em}{-0.4em}

\roottitle{Employment and Education} % Top level section

\vspace{0.5cm}
\headedsection
{\href{https://www.gla.ac.uk/}{\underline{University of Glasgow}}}
{\textsc{Glasgow, UK}}

\headedsubsection
{{Rankin-Sneddon Fellowship}}{Oct 2022 -- Sep 2024}
{\vspace{0.1cm}\bodytext{Independent research position funded by the University of Glasgow, which will begin once I have finished the final year of my Leverhulme Early Career Fellowship.}}

\headedsubsection
{{Early Career Fellowship - Leverhulme Trust}}
{Oct 2021 -- Sep 2022}
{\vspace{0.1cm}\bodytext{Fellowship Title: Resonating Magnetic Field Lines: A Process for Energy Transfer at Earth/Mercury.}}

\vspace{0.5cm}
\headedsection
{\href{https://le.ac.uk/}{\underline{University of Leicester}}}
{\textsc{Leicester, UK}}

\headedsubsection
{{Early Career Fellowship - Leverhulme Trust}}
{Oct 2019 -- Sep 2021}
{\vspace{0.1cm}\bodytext{Fellowship Title (as above)}}

\vspace{0.6cm}
\headedsection % Employer name which can include a hyperlink and location/URL on the right side of the page
{\href{http://www.st-andrews.ac.uk}{\underline{University of St Andrews}}}
{\textsc{St Andrews, UK}}

\headedsubsection % Job title entry for the current employer
{{Post Doctoral Research Assistant}}
{Jun 2016 -- Sep 2019}
{\vspace{0.1cm}\bodytext{Project Title: Synthesis of real and virtual Space Weather data.  \\
\noindent Line Manager: Dr Andrew N. Wright.\\
%\noindent Co-investigator: Dr Michael Hartinger, Virginia Tech University, VA, USA. \\ 
\noindent Funded by the Leverhulme Trust for 3 years.}}

%\headedsubsection % Job title entry for the current employer
%{\acr{Post Doctoral Research Assistant}}
%{Apr '16 -- Jun '16}
%{\vspace{0.1cm}\bodytext{Project: MHD mode coupling in 3D non-Cartesian geometries  \\
%\noindent Line Manager: Dr Andrew N. Wright\\
%\noindent Funded by STFC}}

\headedsubsection % Job title entry for the current employer
{{PhD}}
{Sep 2012 -- Jun 2016}
{\vspace{0.1cm}\bodytext{Thesis title: Numerical Modelling of Ultra Low Frequency Waves in Earth's Magnetosphere. \\
\noindent Thesis supervisor: Dr Andrew N. Wright. \\
\noindent Viva Examiners: Prof David Southwood, Prof Alan Hood. \\
\noindent	Fully funded by STFC.}}

\headedsubsection % Job title entry for the current employer
{MMath (Hons) Mathematics (Fast Track)- 1st Class}
{Sep 2008 -- Jun 2012}
{\vspace{0.1cm}\bodytext{Dissertation title: Structure of Magnetic Separators and 3D Magnetic Neutral Points. \\
Dissertation supervisor: Prof Clare Parnell.}}

\headedsubsection
{Academic Prizes}
{Sep 2008 -- Jun 2012}
{\vspace{0.1cm}\bodytext{Deans' list 2009-2012 for averaging first class grades every year. \\
           Duncan Prize for performance in Applied Mathematics in Senior Honours.}}

\vspace{0.6cm}

\headedsection % Employer name which can include a hyperlink and location/URL on the right side of the page
{\href{http://www.cfa.harvard.edu/opportunities/solar_reu/}{\underline{Harvard Smithsonian Centre for Astrophysics}}}
{\textsc{Cambridge (MA), USA}}

\headedsubsection
{REU (Research Experience for Undergraduates)}
{Jun 2011 -- Aug 2011}
{\vspace{0.1cm} \bodytext{Worked with Dr Steve Saar for 10 weeks as part of the \href{https://www.cfa.harvard.edu/opportunities/solar_reu/}{solar physics REU programme} studying the statistical properties of X-Ray bright points in the solar corona. The research was presented at the 2012 American Astronomical Society Meeting.}}

%------------------------------------------------
\vspace{1.5em}
\spacedhrule{-0.2em}{-0.4em} % Horizontal rule - the first bracket is whitespace before and the second is after








\roottitle{Further Experience}
\vspace{0.2cm}


\headedsubsection
{\underline{Selected Conferences and Courses}}{}

{\vspace{0.1cm} \bodytext{

\bull American Geophysical Union (AGU) Fall Meeting, Virtual, 13-17 Dec 2021 (2 first author posters, 3 co-author posters, 1 co-author presentation).
\vspace{0.1cm}

\bull Autumn Magnetosphere Ionosphere Solar Terrestrial (MIST) UK Meeting, Virtual, 25-26th Nov 2021 (Talk).
\vspace{0.1cm}

\bull Scientific Assembly of IAGA (International Association of Geomagnetism and Aeronomy), hosted virtually by the Indian National Science Academy, 21-27 Aug. (2 Talks)
\vspace{0.1cm}

\bull International Space Science Institute (ISSI) workshop (virtual), Bern, Switzerland, 9-13 Aug 2021. (Talk)
\vspace{0.1cm}

\bull Geospace Environment Modelling (GEM) workshop (virtual), Santa Fe, NM, USA, 25-30 July 2021. (2 Talks)
\vspace{0.1cm}

\bull National Astronomy Meeting, Bath, UK, 19-23 Jul 2021. (Talk) \vspace{0.1cm}


\bull Royal Astronomical Society Specialist Discussion Meeting `Space Weather Energy Pathways and Implications for Impacts', Virtual, 8 Dec 2020 (Poster).
\vspace{0.1cm}

\bull AGU Fall Meeting, Virtual, 1-17 Dec 2020 (Poster).
\vspace{0.1cm}

\bull MIST Autumn Meeting, UK, Virtual, 19-20 Nov 2020. (Poster)
\vspace{0.1cm}

\bull AGU Fall Meeting, San Francisco, USA, 9-13 Dec 2019.
\vspace{0.1cm}

\bull GEM, Santa Fe, NM, USA, 17-23 Jun 2018. (Talk)
\vspace{0.1cm}

\bull MIST Spring Meeting, Southampton, 26-28 Mar 2018. (Talk)

\bull Numerical Techniques in MHD Simulations, Cologne Germany, 16-18 Aug 2017. (Talk) \vspace{0.1cm}

\bull National Astronomy Meeting, Hull, UK, 2-6 Jul 2017. (Talk) \vspace{0.1cm}

\bull Geospace Environment Modelling (GEM), Portsmouth VA, USA, 18-23 Jun 2017. (Talk)
\vspace{0.1cm}

\bull Global Modelling of the Space Weather Chain, Helsinki, Finland, 23-28 Oct 2016. (Talk) \vspace{0.1cm} %

\bull ISSS-12: School for Space Simulations, Prague, Czech Republic 3-10 July 2015. (Poster) \vspace{0.1cm}

\bull BUKS 2015 - MHD Waves and Instabilities, Budapest, Hungary, 25-29 May 2015. (Talk) \vspace{0.1cm}

\bull Geospace: Cluster/MAARBLE/Van Allen Probes, Rhodes, Greece, 15-20 Sep 2014. (Talk) \vspace{0.1cm}

\bull Message-Passing Programming with MPI, Edinburgh Parallel Computing Centre, 2-4 Jul 2014. \vspace{0.1cm}

\bull STFC Advanced Solar Physics Summer School, MSSL (UCL), Surrey, 2nd-6th Sep 2013. \vspace{0.1cm}

%\bull National Astronomy Meeting, St Andrews, 1st-5th Jul 2013. \vspace{0.1cm}

\bull STFC Introductory Solar Physics Summer School, Armagh Observatory, Northern Ireland, 16th-21st Sep 2012. \vspace{0.15cm}


%\href{http://www.arm.ac.uk/summerschool2012/}{STFC introductory summer school on solar physics}, Armagh Observatory, 16th-21st Sep '12 \vspace{0.1cm}

%           \href{https://www.ras.org.uk/events-and-meetings/ras-meetings/nam2013}{National Astronomy Meeting}, St Andrews, 1st-5th Jul '13 \vspace{0.1cm}
					
%					 STFC advanced summer school on Solar and Solar-Terrestrial physics, MSSL, 2nd-6th Sep '13 \vspace{0.1cm}
					
%					 \href{https://www.epcc.ed.ac.uk/education-training/general-training/course-portfolio/course-1}{Message-Passing Programming with MPI}, Edinburgh Parallel Computing Centre (EPCC), 2nd-4th July '14 \vspace{0.1cm}
					
					 
					%\href{http://www5.noa.gr/images/presentations/saturday/AM2/Elsden\%20-\%20The\%20Use\%20of\%20the\%20Poynting\%20Vector\%20in\%20Interpreting\%20ULF\%20waves.pdf}{(Talk)} \vspace{0.1cm}
					
					
					
					
					
					%\href{http://www.mist.ac.uk/meetings/upcoming-meetings/149-autumn-mist-2015-first-announcement}{Autumn MIST UK}, London, 27th November '15. \href{http://www-solar.mcs.st-and.ac.uk/~telsden/talks/Elsden_MIST_15.pdf}{(Talk)} \vspace{0.1cm}
					
					%\href{http://spaceweatherchain2016.aalto.fi/}{Global Modelling of the Space Weather Chain}, Helsinki, 23rd-28th October '16. \href{http://www-solar.mcs.st-and.ac.uk/~telsden/talks/Elsden_talk_GMSWC16.pdf}{(Talk)} \vspace{0.1cm} %
					
					%\href{}{Autumn MIST UK}, London, November '16. \vspace{0.1cm}
					
					%\href{}{GEM Workshop}, Viriginia, VA, USA, 18th-23rd June '17. \href{}{(Talk)} \vspace{0.1cm}
					
					%\href{}{National Astronomy Meeting}, Hull, UK, 2nd-6th July '17. \href{}{(Talk)} \vspace{0.1cm}
					
					}}
					

					
%\headedsubsection
%{Other Talks}{}
%\vspace{0.1cm}
%{\vspace{0.1cm}\bodytext{\href{http://www-solar.mcs.st-andrews.ac.uk/}{St Andrews Solar and Magnetospheric theory group} seminar - Jun '13 \href{http://www-solar.mcs.st-and.ac.uk/~telsden/talks/Solar_Seminar_1.pdf}{(Talk)}, Jun '14 \href{http://www-solar.mcs.st-and.ac.uk/~telsden/talks/Solar_Seminar_2.pdf}{(Talk)}, March '15 \href{http://www-solar.mcs.st-and.ac.uk/~telsden/talks/Solar_Seminar_3.pdf}{(Talk)}, Apr '16 \href{}{(Talk)}, Feb '17 \href{}{(Talk)}   \vspace{0.1cm}

%Research in the UK afternoon, Cambridge university - Nov '13, \href{http://www-solar.mcs.st-and.ac.uk/~telsden/talks/Cambridge13.pdf}{(Talk)}  \vspace{0.1cm}

%School of Mathematics and Statistics Research Day, St Andrews - 20th Jan '15 \href{http://www-solar.mcs.st-and.ac.uk/~telsden/talks/telsden_rsch_day_Jan15.pdf}{(Talk)} \vspace{0.1cm}

%Postgraduate Interdisciplinary Mathematics Symposium, The Burn, Edzell, Scotland, 26th-28th Jan '15 \href{http://www-solar.mcs.st-and.ac.uk/~telsden/talks/telsden_burn15.pdf}{(Talk)}. \vspace{0.1cm}

%Applied Postgraduate Seminar, St Andrews, 3rd Dec '15 \href{http://www-solar.mcs.st-and.ac.uk/~telsden/talks/APS_talk.pdf}{(Talk)}. 


%}}
\headedsubsection
{\underline{Computing}}{}

{\vspace{0.1cm} \bodytext{

\bull Proficient in Fortran 90, Interactive Data Language (IDL), Message Passing Interface (MPI) parallelisation \vspace{0.1cm}, Linux operating system and LaTex.

\bull Experience with Python (completed online course of \href{http://scipy-lectures.org/index.html}{SciPy lectures}) and Maple. \vspace{0.1cm}

\bull Experience with running large scale numerical simulations on large parallel architectures. 

}}
%\hspace{-0.1cm}\bull Have developed my own codes in Fortran utilising MPI parallelisation to solve the linear magnetohydrodynamic equations. \vspace{0.25cm} }}
				
					%the linear magnetohydrodynamic equations.
		%Expand this section, perhaps listing specific undergrad courses taken in fortran and types of codes i've written in Fortran.
		



%\headedsubsection{Undergraduate Tutor - University of St Andrews}
%{Sep '12 -- Present}
%{\vspace{0.1cm} \bodytext{Tutor for undergraduate courses in mathematics including MT2001 - Mathematics, MT2003 - Applied Mathematics, \href{http://www.st-andrews.ac.uk/maths/current/ug/modules/4000-level/\#MT4112}{MT4112 - Computing in Mathematics}, MT2000 - Computing with Python and \href{http://www.st-andrews.ac.uk/maths/current/ug/modules/2000-level/\#MT2503}{MT2503 - Multivariate Calculus.}

%\vspace{0.2cm}

%These courses involved a variety of formats e.g. small group tutorials ($<$10 students), large group examples classes ($\sim$ 50 students) and computing labs.}}



%\bull European Solar Physics Online Seminar (EPSOS) - May 2017. 
%\vspace{0.1cm}

		
\headedsubsection
{\underline{Public Outreach}}{}

{\vspace{0.1cm} 

\bodytext{
\bull Lecture at Perth Grammar School, Perth, 26th Oct 2018, discussing mathematics at University to final year pupils. \vspace{0.1cm} 

\bull Lecture at St Ninian's High School, Glasgow, 12th Sep 2018, titled 'Applied Mathematics in Space/ What can you do with Maths?'. Over 300 5th and 6th year mathematics pupils attended. \vspace{0.1cm} 

\bull Organised a workshop at a STEM Fair at St Ninian's High School, Glasgow, 13th Mar 2018, discussing mathematics with high school pupils. \vspace{0.1cm} 

\bull William Bright Society Lecture, Glenalmond College, Perthshire, Scotland, 22nd Nov 2017. This is the school's academic enrichment programme, the lecture was titled 'Applied Mathematics in Space' and was attended by around 80 high school pupils aged 11-17. \vspace{0.1cm} 

\bull Developed and organised a `maths busking' event as part of Maths Week Scotland, 11-17th Sep 2017. This involved designing mathematical puzzles and taking them to the streets of St Andrews and engaging members of the public. \vspace{0.1cm} 

\bull Participated in and gave a lecture at the Highland maths weekend at Lagganlia, Scotland in November 2016. This involved around 50 final year high school students studying mathematics, listening to and interacting with experts in various STEM subjects. \vspace{0.1cm} 

% \href{http://www-solar.mcs.st-and.ac.uk/~telsden/talks/Elsden_AdvHM_2016.pdf}{(talk)} \vspace{0.1cm}

%\bull Regular attendee of Science Discovery Days at St Andrews University. These days involve interacting with families from the surrounding community about the research that happens at the university. }} 
}}



\vspace{1.5em}
\spacedhrule{-0.2em}{-0.4em} % Horizontal rule - the first bracket is whitespace before and the second is after

\roottitle{References}

{\vspace{0.2cm} 

\bodytext{
\bull Prof Tim Yeoman, School of Physics and Astronomy, University of Leicester, University Road, Leicester, UK, LE1 7RH, email: yxo@leicester.ac.uk (Previous line manager at Leicester)

\vspace{0.1cm} 
\bull Dr Andrew Wright, School of Mathematics and Statistics, University of St Andrews, North Haugh, St Andrews, UK, KY16 9SS, email: anw@st-andrews.ac.uk (PhD supervisor and previous line manager at St Andrews)					
					
\vspace{0.1cm} 
\bull Prof Ian Strachan, Head of School of Mathematics and Statistics, University of Glasgow, University Place, Glasgow G12 8QQ, email: Ian.Strachan@glasgow.ac.uk (Current line manager at Glasgow)

%\bull Prof David Southwood, Blackett Laboratory, Imperial College London, London, SW7 2BW UK, email: d.southwood@imperial.ac.uk


%\vspace{0.1cm} 
%\bull Prof Jonathan Rae, Department of Mathematics, Physics and Electrical Engineering, Northumbria University, Ellison Place, Newcastle upon Tyne, UK, NE1 8ST, email: jonathan.rae@northumbria.ac.uk

}}



					
					
					
					
					
					
					
					
					
					
					
					

%----------------------------------------------------------------------------------------

